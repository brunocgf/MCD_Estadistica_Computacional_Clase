\documentclass[]{article}
\usepackage{lmodern}
\usepackage{amssymb,amsmath}
\usepackage{ifxetex,ifluatex}
\usepackage{fixltx2e} % provides \textsubscript
\ifnum 0\ifxetex 1\fi\ifluatex 1\fi=0 % if pdftex
  \usepackage[T1]{fontenc}
  \usepackage[utf8]{inputenc}
\else % if luatex or xelatex
  \ifxetex
    \usepackage{mathspec}
  \else
    \usepackage{fontspec}
  \fi
  \defaultfontfeatures{Ligatures=TeX,Scale=MatchLowercase}
\fi
% use upquote if available, for straight quotes in verbatim environments
\IfFileExists{upquote.sty}{\usepackage{upquote}}{}
% use microtype if available
\IfFileExists{microtype.sty}{%
\usepackage{microtype}
\UseMicrotypeSet[protrusion]{basicmath} % disable protrusion for tt fonts
}{}
\usepackage[margin=1in]{geometry}
\usepackage{hyperref}
\hypersetup{unicode=true,
            pdftitle={EstComp-Tarea11},
            pdfauthor={Bruno Gonzalez},
            pdfborder={0 0 0},
            breaklinks=true}
\urlstyle{same}  % don't use monospace font for urls
\usepackage{color}
\usepackage{fancyvrb}
\newcommand{\VerbBar}{|}
\newcommand{\VERB}{\Verb[commandchars=\\\{\}]}
\DefineVerbatimEnvironment{Highlighting}{Verbatim}{commandchars=\\\{\}}
% Add ',fontsize=\small' for more characters per line
\usepackage{framed}
\definecolor{shadecolor}{RGB}{248,248,248}
\newenvironment{Shaded}{\begin{snugshade}}{\end{snugshade}}
\newcommand{\AlertTok}[1]{\textcolor[rgb]{0.94,0.16,0.16}{#1}}
\newcommand{\AnnotationTok}[1]{\textcolor[rgb]{0.56,0.35,0.01}{\textbf{\textit{#1}}}}
\newcommand{\AttributeTok}[1]{\textcolor[rgb]{0.77,0.63,0.00}{#1}}
\newcommand{\BaseNTok}[1]{\textcolor[rgb]{0.00,0.00,0.81}{#1}}
\newcommand{\BuiltInTok}[1]{#1}
\newcommand{\CharTok}[1]{\textcolor[rgb]{0.31,0.60,0.02}{#1}}
\newcommand{\CommentTok}[1]{\textcolor[rgb]{0.56,0.35,0.01}{\textit{#1}}}
\newcommand{\CommentVarTok}[1]{\textcolor[rgb]{0.56,0.35,0.01}{\textbf{\textit{#1}}}}
\newcommand{\ConstantTok}[1]{\textcolor[rgb]{0.00,0.00,0.00}{#1}}
\newcommand{\ControlFlowTok}[1]{\textcolor[rgb]{0.13,0.29,0.53}{\textbf{#1}}}
\newcommand{\DataTypeTok}[1]{\textcolor[rgb]{0.13,0.29,0.53}{#1}}
\newcommand{\DecValTok}[1]{\textcolor[rgb]{0.00,0.00,0.81}{#1}}
\newcommand{\DocumentationTok}[1]{\textcolor[rgb]{0.56,0.35,0.01}{\textbf{\textit{#1}}}}
\newcommand{\ErrorTok}[1]{\textcolor[rgb]{0.64,0.00,0.00}{\textbf{#1}}}
\newcommand{\ExtensionTok}[1]{#1}
\newcommand{\FloatTok}[1]{\textcolor[rgb]{0.00,0.00,0.81}{#1}}
\newcommand{\FunctionTok}[1]{\textcolor[rgb]{0.00,0.00,0.00}{#1}}
\newcommand{\ImportTok}[1]{#1}
\newcommand{\InformationTok}[1]{\textcolor[rgb]{0.56,0.35,0.01}{\textbf{\textit{#1}}}}
\newcommand{\KeywordTok}[1]{\textcolor[rgb]{0.13,0.29,0.53}{\textbf{#1}}}
\newcommand{\NormalTok}[1]{#1}
\newcommand{\OperatorTok}[1]{\textcolor[rgb]{0.81,0.36,0.00}{\textbf{#1}}}
\newcommand{\OtherTok}[1]{\textcolor[rgb]{0.56,0.35,0.01}{#1}}
\newcommand{\PreprocessorTok}[1]{\textcolor[rgb]{0.56,0.35,0.01}{\textit{#1}}}
\newcommand{\RegionMarkerTok}[1]{#1}
\newcommand{\SpecialCharTok}[1]{\textcolor[rgb]{0.00,0.00,0.00}{#1}}
\newcommand{\SpecialStringTok}[1]{\textcolor[rgb]{0.31,0.60,0.02}{#1}}
\newcommand{\StringTok}[1]{\textcolor[rgb]{0.31,0.60,0.02}{#1}}
\newcommand{\VariableTok}[1]{\textcolor[rgb]{0.00,0.00,0.00}{#1}}
\newcommand{\VerbatimStringTok}[1]{\textcolor[rgb]{0.31,0.60,0.02}{#1}}
\newcommand{\WarningTok}[1]{\textcolor[rgb]{0.56,0.35,0.01}{\textbf{\textit{#1}}}}
\usepackage{graphicx,grffile}
\makeatletter
\def\maxwidth{\ifdim\Gin@nat@width>\linewidth\linewidth\else\Gin@nat@width\fi}
\def\maxheight{\ifdim\Gin@nat@height>\textheight\textheight\else\Gin@nat@height\fi}
\makeatother
% Scale images if necessary, so that they will not overflow the page
% margins by default, and it is still possible to overwrite the defaults
% using explicit options in \includegraphics[width, height, ...]{}
\setkeys{Gin}{width=\maxwidth,height=\maxheight,keepaspectratio}
\IfFileExists{parskip.sty}{%
\usepackage{parskip}
}{% else
\setlength{\parindent}{0pt}
\setlength{\parskip}{6pt plus 2pt minus 1pt}
}
\setlength{\emergencystretch}{3em}  % prevent overfull lines
\providecommand{\tightlist}{%
  \setlength{\itemsep}{0pt}\setlength{\parskip}{0pt}}
\setcounter{secnumdepth}{0}
% Redefines (sub)paragraphs to behave more like sections
\ifx\paragraph\undefined\else
\let\oldparagraph\paragraph
\renewcommand{\paragraph}[1]{\oldparagraph{#1}\mbox{}}
\fi
\ifx\subparagraph\undefined\else
\let\oldsubparagraph\subparagraph
\renewcommand{\subparagraph}[1]{\oldsubparagraph{#1}\mbox{}}
\fi

%%% Use protect on footnotes to avoid problems with footnotes in titles
\let\rmarkdownfootnote\footnote%
\def\footnote{\protect\rmarkdownfootnote}

%%% Change title format to be more compact
\usepackage{titling}

% Create subtitle command for use in maketitle
\providecommand{\subtitle}[1]{
  \posttitle{
    \begin{center}\large#1\end{center}
    }
}

\setlength{\droptitle}{-2em}

  \title{EstComp-Tarea11}
    \pretitle{\vspace{\droptitle}\centering\huge}
  \posttitle{\par}
    \author{Bruno Gonzalez}
    \preauthor{\centering\large\emph}
  \postauthor{\par}
      \predate{\centering\large\emph}
  \postdate{\par}
    \date{7/11/2019}


\begin{document}
\maketitle

\hypertarget{familias-conjugadas}{%
\subsection{11-Familias conjugadas}\label{familias-conjugadas}}

\hypertarget{modelo-beta-binomial}{%
\paragraph{1. Modelo Beta-Binomial}\label{modelo-beta-binomial}}

Una compañía farmacéutica afirma que su nueva medicina incrementa la
probabilidad de concebir un niño (sexo masculino), pero aún no publican
estudios. Supón que conduces un experimento en el cual \(50\) parejas se
seleccionan de manera aleatoria de la población, toman la medicina y
conciben.

\begin{enumerate}
\def\labelenumi{\alph{enumi})}
\tightlist
\item
  Quieres estimar la probabilidad de concebir un niño para parejas que
  toman la medicina. ¿Cuál es una inicial apropiada? No tiene que estar
  centrada en \(0.5\) pues esta corresponde a personas que no toman la
  medicina, y la inicial debe reflejar tu incertidumbre sobre el efecto
  de la droga.
\end{enumerate}

Dado que tenemos el número de la muestra \(n = 50\), podemos partir
definiendo a \(m\), la proporción de éxitos (concebir un niño) y
establecerla como \(m = 0.7\). De esta manera: \[ \begin{align*}
&a = mn = 35 \\
&b = (1-m)n = 15
\end{align*}\]

Así, podemos definir la distribución a priori de la siguiente manera:

\begin{Shaded}
\begin{Highlighting}[]
\NormalTok{N <-}\StringTok{  }\DecValTok{50}
\NormalTok{m <-}\StringTok{  }\FloatTok{0.7}
\NormalTok{a <-}\StringTok{  }\NormalTok{m}\OperatorTok{*}\NormalTok{N}
\NormalTok{b <-}\StringTok{  }\NormalTok{(}\DecValTok{1}\OperatorTok{-}\NormalTok{m)}\OperatorTok{*}\NormalTok{N}

\NormalTok{priori <-}\StringTok{ }\KeywordTok{ggplot}\NormalTok{(}\KeywordTok{tibble}\NormalTok{(}\DataTypeTok{x =} \KeywordTok{c}\NormalTok{(}\DecValTok{0}\NormalTok{,}\DecValTok{1}\NormalTok{)), }\KeywordTok{aes}\NormalTok{(x)) }\OperatorTok{+}
\StringTok{  }\KeywordTok{stat_function}\NormalTok{(}\DataTypeTok{fun =}\NormalTok{ dbeta, }\DataTypeTok{args =} \KeywordTok{list}\NormalTok{(a,b)) }\OperatorTok{+}
\StringTok{  }\KeywordTok{labs}\NormalTok{(}\DataTypeTok{title =} \StringTok{"A priori"}\NormalTok{,}
    \DataTypeTok{x =} \StringTok{""}\NormalTok{,}
    \DataTypeTok{y =} \StringTok{""}\NormalTok{)}

\NormalTok{priori}
\end{Highlighting}
\end{Shaded}

\includegraphics{EstComp-Tarea11-BCG_files/figure-latex/1_priori-1.pdf}

\begin{enumerate}
\def\labelenumi{\alph{enumi})}
\setcounter{enumi}{1}
\tightlist
\item
  Usando tu inicial de a) grafica la posterior y decide si es creíble
  que las parejas que toman la medicina tienen una probabilidad de
  \(0.5\) de concebir un niño.
\end{enumerate}

Sea \(z\) el número de éxitos de la prueba, en esta caso la gráfica
posterior es de la siguiente manera:

\begin{Shaded}
\begin{Highlighting}[]
\CommentTok{# Supongamos z}
\NormalTok{z =}\StringTok{ }\DecValTok{30}

\NormalTok{post <-}\StringTok{ }\KeywordTok{ggplot}\NormalTok{(}\KeywordTok{tibble}\NormalTok{(}\DataTypeTok{x =} \KeywordTok{c}\NormalTok{(}\DecValTok{0}\NormalTok{,}\DecValTok{1}\NormalTok{)), }\KeywordTok{aes}\NormalTok{(x)) }\OperatorTok{+}
\StringTok{  }\KeywordTok{stat_function}\NormalTok{(}\DataTypeTok{fun =}\NormalTok{ dbeta, }\DataTypeTok{args =} \KeywordTok{list}\NormalTok{(z}\OperatorTok{+}\NormalTok{a,N}\OperatorTok{-}\NormalTok{z}\OperatorTok{+}\NormalTok{b)) }\OperatorTok{+}
\StringTok{  }\KeywordTok{labs}\NormalTok{(}\DataTypeTok{title =} \StringTok{"A posteriori"}\NormalTok{,}
    \DataTypeTok{x =} \StringTok{""}\NormalTok{,}
    \DataTypeTok{y =} \StringTok{""}\NormalTok{)}

\NormalTok{post}
\end{Highlighting}
\end{Shaded}

\includegraphics{EstComp-Tarea11-BCG_files/figure-latex/1_post-1.pdf}

Dado el supuesto de \(z = 60\), sí es probable que la medicina surja
efecto ya la probabilidad de éxito es mayormente mayor a \emph{50\%}.

\begin{enumerate}
\def\labelenumi{\alph{enumi})}
\setcounter{enumi}{2}
\tightlist
\item
  Supón que la farmacéutica asevera que la probabilidad de concebir un
  niño cuando se toma la medicina es cercana al \(60\%\) con alta
  certeza. Representa esta postura con una distribución inicial
  \(Beta(60,40)\). Comparala con la inicial de un escéptico que afirma
  que la medicina no hace diferencia, representa esta creencia con una
  inicial \(Beta(50,50)\). ¿Cómo se compara la probabilidad posterior de
  concebir un niño (usando las distintas iniciales)?
\end{enumerate}

En esta caso, las distribuciones a priori se distribuyen de la siguiente
manera:

\begin{Shaded}
\begin{Highlighting}[]
\NormalTok{priori_comp1 <-}\StringTok{ }\KeywordTok{ggplot}\NormalTok{(}\KeywordTok{tibble}\NormalTok{(}\DataTypeTok{x =} \KeywordTok{c}\NormalTok{(}\DecValTok{0}\NormalTok{,}\DecValTok{1}\NormalTok{)), }\KeywordTok{aes}\NormalTok{(x)) }\OperatorTok{+}
\StringTok{  }\KeywordTok{stat_function}\NormalTok{(}\DataTypeTok{fun =}\NormalTok{ dbeta, }\DataTypeTok{args =} \KeywordTok{list}\NormalTok{(}\DecValTok{60}\NormalTok{,}\DecValTok{40}\NormalTok{)) }\OperatorTok{+}
\StringTok{  }\KeywordTok{labs}\NormalTok{(}\DataTypeTok{title =} \StringTok{"Beta(60,40)"}\NormalTok{,}
    \DataTypeTok{x =} \StringTok{""}\NormalTok{,}
    \DataTypeTok{y =} \StringTok{""}\NormalTok{)}

\NormalTok{priori_comp2 <-}\StringTok{ }\KeywordTok{ggplot}\NormalTok{(}\KeywordTok{tibble}\NormalTok{(}\DataTypeTok{x =} \KeywordTok{c}\NormalTok{(}\DecValTok{0}\NormalTok{,}\DecValTok{1}\NormalTok{)), }\KeywordTok{aes}\NormalTok{(x)) }\OperatorTok{+}
\StringTok{  }\KeywordTok{stat_function}\NormalTok{(}\DataTypeTok{fun =}\NormalTok{ dbeta, }\DataTypeTok{args =} \KeywordTok{list}\NormalTok{(}\DecValTok{50}\NormalTok{,}\DecValTok{50}\NormalTok{)) }\OperatorTok{+}
\StringTok{  }\KeywordTok{labs}\NormalTok{(}\DataTypeTok{title =} \StringTok{"Beta(50,50)"}\NormalTok{,}
    \DataTypeTok{x =} \StringTok{""}\NormalTok{,}
    \DataTypeTok{y =} \StringTok{""}\NormalTok{)}

\KeywordTok{grid.arrange}\NormalTok{(priori_comp1, priori_comp2, }\DataTypeTok{nrow =} \DecValTok{1}\NormalTok{)}
\end{Highlighting}
\end{Shaded}

\includegraphics{EstComp-Tarea11-BCG_files/figure-latex/1_priori_comp-1.pdf}

Usando el mismo número de éxitos del inciso b), tenemos las siguientes
distribuciones a posteriori.

\begin{Shaded}
\begin{Highlighting}[]
\NormalTok{post_comp1 <-}\StringTok{ }\KeywordTok{ggplot}\NormalTok{(}\KeywordTok{tibble}\NormalTok{(}\DataTypeTok{x =} \KeywordTok{c}\NormalTok{(}\DecValTok{0}\NormalTok{,}\DecValTok{1}\NormalTok{)), }\KeywordTok{aes}\NormalTok{(x)) }\OperatorTok{+}
\StringTok{  }\KeywordTok{stat_function}\NormalTok{(}\DataTypeTok{fun =}\NormalTok{ dbeta, }\DataTypeTok{args =} \KeywordTok{list}\NormalTok{(z }\OperatorTok{+}\StringTok{ }\DecValTok{60}\NormalTok{,N }\OperatorTok{-}\StringTok{ }\NormalTok{z }\OperatorTok{+}\DecValTok{40}\NormalTok{)) }\OperatorTok{+}
\StringTok{  }\KeywordTok{labs}\NormalTok{(}\DataTypeTok{title =} \StringTok{"Beta(60,40)"}\NormalTok{,}
    \DataTypeTok{x =} \StringTok{""}\NormalTok{,}
    \DataTypeTok{y =} \StringTok{""}\NormalTok{)}

\NormalTok{post_comp2 <-}\StringTok{ }\KeywordTok{ggplot}\NormalTok{(}\KeywordTok{tibble}\NormalTok{(}\DataTypeTok{x =} \KeywordTok{c}\NormalTok{(}\DecValTok{0}\NormalTok{,}\DecValTok{1}\NormalTok{)), }\KeywordTok{aes}\NormalTok{(x)) }\OperatorTok{+}
\StringTok{  }\KeywordTok{stat_function}\NormalTok{(}\DataTypeTok{fun =}\NormalTok{ dbeta, }\DataTypeTok{args =} \KeywordTok{list}\NormalTok{(z }\OperatorTok{+}\StringTok{ }\DecValTok{50}\NormalTok{,N }\OperatorTok{-}\StringTok{ }\NormalTok{z }\OperatorTok{+}\DecValTok{50}\NormalTok{)) }\OperatorTok{+}
\StringTok{  }\KeywordTok{labs}\NormalTok{(}\DataTypeTok{title =} \StringTok{"Beta(50,50)"}\NormalTok{,}
    \DataTypeTok{x =} \StringTok{""}\NormalTok{,}
    \DataTypeTok{y =} \StringTok{""}\NormalTok{)}

\KeywordTok{grid.arrange}\NormalTok{(post_comp1, post_comp2, }\DataTypeTok{nrow =} \DecValTok{1}\NormalTok{)}
\end{Highlighting}
\end{Shaded}

\includegraphics{EstComp-Tarea11-BCG_files/figure-latex/1_post_comp-1.pdf}

La distribución posterior para un escéptico sigue representando su
escepticismo, lo ilustra que las distribuicones a priori impactan las
posteriori independientemente de la información disponible.

\hypertarget{otra-familia-conjugada}{%
\paragraph{2. Otra familia conjugada}\label{otra-familia-conjugada}}


\end{document}
